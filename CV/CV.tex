\documentclass[11pt,letterpaper,sans]{moderncv}

\moderncvstyle{banking} % 'casual', 'classic', 'oldstyle' 'banking', 'fancy'
\moderncvcolor{blue} % 'blue' (default), 'orange', 'green', 'red', 'purple', 'grey' and 'black'
\usepackage{changepage}
\usepackage{enumitem} %itemize customization
\usepackage[scale=0.75]{geometry} % Reduce document margins
\usepackage[inkscapeformat=png]{svg}
\addtolength{\voffset}{-3.5em}
\addtolength{\textheight}{7em}
\setlength{\hintscolumnwidth}{2.5cm} % width of dates column

%\quote{``It's hard to do a really good job
%on anything you don't think about in the shower.''
%--- Paul Graham}
\newcommand{\Rlogo}{\protect\includegraphics[height=1.8ex,keepaspectratio]{../files/Rlogo.png}}
\newcommand{\Jupyter}{\protect\includesvg[width=0.15in]{../files/Jupyter_logo.svg}}
\newcommand{\entry}[6]%
{\cventry[0.85em]{#4}{#3}{#2}{#1}{#5}{#6}} % banking
\newcommand{\cvurl}[1]{\bgroup\color{blue}\normalfont\url{#1}\egroup}
\newcommand{\arxiv}[1]{\bgroup\color{blue}\normalfont\rmfamily%
\href{http://arxiv.org/abs/#1}{arXiv:#1}\egroup}



\lfoot{\emph{Last update: \today}}

\newcommand{\subtext}[1]{\bgroup\footnotesize\rmfamily\color{black!70}#1\egroup}
\newcommand{\previous}[1]{\bgroup\color{black}\bfseries#1\egroup}

\newcommand{\job}[4]{\smallskip\cvitemwithcomment{#2}{#1}{#3}%
\unskip\subtext{#4}\medskip}

\newcommand{\pub}[2]{\cvlistitem{#1. \\ \subtext{#2}}\smallskip}
\newcommand{\paper}[3]{\pub{#1}{#2. \arxiv{#3}.}}

\newcommand{\cventrywithsubtext}[7]{\cventry{#1}{#2}{#3}{#4}{#5}{#6}%
\unskip\subtext{#7}\medskip}

\name{Cristian}{Curaba}
\email{cristiancuraba00@gmail.com}
%social networks
\social[linkedin]{Cristian-Curaba}
\social[github]{Cristian-Curaba}

\address{Via Gemona 92, Udine, 33100}{Italy}
\lfoot{\emph{Last updated \today}}

\photo[75pt]{../files/Foto_cv.png}

\patchcmd{\makehead}% <cmd>
  {\\[2.5em]}% <search>
  {\hfill\raisebox{-1.9cm}[0pt][0pt]{\includegraphics[width=.14\textwidth]{../files/profile-pic.png}}\\[2.0em]}% <replace>
  {}{}% <success><failure>

\begin{document}

\makecvtitle % Print the CV title
\vspace*{-1.5em}

\section{About me}
\begin{adjustwidth}{0cm}{0cm}
A passionate mathematician and coder dedicated to driving impactful research through technical expertise. With an eclectic, creative approach, I excel in collaborative environments, prioritizing trustworthiness, and intellectual rigor.
\end{adjustwidth}

\section{Education}

\cventry{2022 -- 2024}{Data Science and Scientific Computing}{Master's Degree at University of Trieste, Italy}{}{Grade: 110/110 cum Laude}{Thesis: \textit{Integrating Large Language Models and Formal Verification for Automated Cryptographic Protocol Vulnerability Detection}}


\cventry{2019 -- 2024}{Student of the Scientific Class}{School for Advanced Studies of Udine -- Second-level Master's degree}{}{Expected Grade: 110/110 cum Laude}{\small The School for Advanced Studies of the University of Udine "Di Toppo Wassermann" is a higher learning institution based on merit. After a selective admission, provides a five-year scholarship covering tuition, board, and lodging. During this period, students have to attend extra courses and exams, culminating in a second-level Master's degree diploma. Learn more at \cvurl{https://superiore.uniud.it/en}.}


\cventry{2019 -- 2022}{Mathematics}{Bachelor's Degree at University of Udine, Italy}{}{Grade: 110/110}
{Thesis:\textit{Teoria descrittiva della complessità: la logica FP+C cattura Ptime nella classe dei grafi ad intervallo}}

\section{Work Experience}

\cventry{2024 (6 months)}{AI Safety Research}{Apart Fellowship }{}{}{}
\cvitem{}{Developed \textbf{CryptoFormalEval}: a benchmark aimed at assessing the capability of LLMs to detect and analyze vulnerabilities in cryptographic protocols.}

\cventry{2023 -- 2024 (4 months)}{Development and Research }{SMS Group}{}{}{}
\cvitem{}{Applied machine learning methods and data analysis to devise strategies for controlling electric furnaces. \\}

\section{Computer and Programming Skills}
\cvitem{Advanced}{\textsc{C}, \LaTeX, C\texttt{++}, \textsc{Python} (\textsc{PyTorch})}
\cvitem{Intermediate}{\Rlogo,  \textsc{Shell Unix}, \textsc{PostgreSQL}, \textsc{Git}}
\cvitem{Basic}{\textsc{Pyro}, \textsc{OpenMP, MPI},  \textsc{HTML}, \textsc{Matlab}, \textsc{Tamarin}}

\section{Languages}
\cvitemwithcomment{Italian}{Mother tongue}{}
\cvitemwithcomment{English}{Proficient user (C1)}{\href{http://ielts.ucles.org.uk}{IELTS}}

\section{Experiences}
\pub{AI Act Summer School (2024)}{Participated in the AI Act Summer School at University of Udine, a 3-day program on AI ethics, AI policy, and AI governance.}
\pub{Alignment Mapping Program (2024)}{Successfully completed the Alignment Mapping Program, an 8-week program on AGI issues, AGI plans, and personal career development.}
\pub{ML4Good French (2024)}{Participation in the 10-day bootcamp program on Machine Learning for AGI Safety run by EffiSciences.}
\pub{Testing Language Models for Autonomous Capabilities - Apart Hackathon}{Winner as best Quality Assurance Tester. Evaluated Task: Tamarin Formalizer.}
\pub{AI Safety North course (2023)}{Fundamentals of Artificial General Intelligence (about 40 hours). Shared insights in a weekly-planned discussion group.}
\pub{International Collegiate Programming Contest (SWERC - 2023/2024)}{Participation in the international programming competition (C++) in the team of The School for Advanced Studies "Di Toppo Wassermann". Classified 55/103.}
\pub{Sisifo Association (2020 -- 2024)}{Actively participating in a non-profit association mainly for divulgation. \href{https://www.sisifoassociazione.it/}{Website link}}

\section{Publications}

\cvitem{[1]}{\textbf{Cristian Curaba} and Denis D'Ambrosi and Alessandro Minisini and Natalia Pérez-Campanero Antolín. \textit{CryptoFormalEval: Integrating LLMs and Formal Verification for Automated Cryptographic Protocol Vulnerability Detection}. \\
\textbf{NeurIPS 2024 Workshop}: \href{https://openreview.net/forum?id=hqvJT2pOxu}{Sys2-Reasoning Poster} \\
\textbf{ArXiv}: \href{https://www.arxiv.org/abs/2411.13627}{arXiv:2411.13627}}

\section{Projects}
\cvitem{CryptoFormalEval}{Introduced a benchmark for testing how well LLMs can find vulnerabilities in cryptographic protocols. By combining LLMs with symbolic reasoning tools like Tamarin, we aim to improve the efficiency and thoroughness of protocol analysis, paving the way for future AI-powered cybersecurity defenses. \\
Link: \href{https://github.com/Cristian-Curaba/CryptoFormalEval}{https://github.com/Cristian-Curaba/CryptoFormalEval}}

\cvitem{Linearization of CNN Layer}{Built a custom loss function to penalize non-linearities through the convolutional layers (PyTorch package). Achieved an interesting outcome: the ReLU activation function arises naturally in a parametrized family of functions. \\
Link: \href{https://github.com/Cristian-Curaba/Linearization-of-CNN-layer}{https://github.com/Cristian-Curaba/Linearization-of-CNN-layer}}
\subsection{Interests}
\cvlistdoubleitem{Educational videos}{Board and card games}
\cvlistdoubleitem{Sports}{Reading}

\vfill
\emptysection{}\closesection
\vfill
\begin{center}
\textit{\small I hereby authorize the use of my personal data in accordance with the GDPR 679/16}
\end{center}

\end{document}
